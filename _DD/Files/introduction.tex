\subsection{Purpose}
The Design Document has the main goal of illustrate a more technical and specific indications to help developers implement what RASD describes.

In particular, where the previous mentioned document is more focused on the requirements, goals, general context description and their theoretical correctness, DD defines main guidelines to concrete those ideas and principles, 
making the program real.\\
\\
The procedures and phases here reported are, namely, implementation, integration and testing plans.

Finally, is reported the singular effort spent to realize the Document.

\subsection{Scope}
CodeKataBattle is an application thought to train coding skills using the "kata Battle" principles. 

This technique, typical of martial arts, consists of repeating an action to improve it, until a good level of confidence, correctness and precision is achieved.\\
\\
In this context, the application provides two different type of user that will interact with it, which are Students and Educators.

Educator is the one who oversee the Students' work and learning: he/she designs Tournaments and their corresponding Battles, which are the last online arenas in which Students compete to develop a particular code. Furthermore, 
he/she specify potential badges that could be given to the Students who deserve them the most during each Tournament.

Instead, the Student is a user who subscribes to CKB in order to advance his or her abilities in software development; as a result, he or she may participate in Tournaments alone or in Teams with other Students. Through challenging 
other colleagues, he or she will gain experience in developing and solving real-world programming problems.\\
\\
The Platform will assist the Educator in overseeing Tournaments and Battles, and it will automatically evaluate, based on certain standards, the code that the Students provide. Furthermore, it will facilitate Students' learning by 
simplifying the user interface and making it simple to peruse the list of all accessible Tournaments, distinctly outlining the prerequisites and goals of each of them.\\
\\
The app's services are accessible through a Web App, that you reach through browser, and among them are included the creation, for Educators, and registration, for Students, to Tournaments and Battles, code evaluation, creation of Teams of Students and invitations 
to collaborate to a Tournament management for Educators. 
The architecture of the product is divided in 4 tiers: client, web server, application server and data server. Three layer are adopted: presentation layer, application layer and data layer.
The app's services described before are implemented as microservices: there are microservices that operate to a correct functioning of the managing of the user's account and microservices to managed
Battles and Tournaments with everything is concerned with them.
\subsection{Definitions, acronyms, abbreviations}
To better understand the Document and the Actors involved here are reported some Definitions.
\begin{enumerate}[label=$\bullet$]
    \item \textbf{Platform}\\Represents the CodeKataBattle System's behavior as a whole, describes its workflows and interactions.
    \item \textbf{User}\\With User is intended, depending on the situation, Educator or Student. The entity, in particular will model behaviors and interactions common to both main Actors.
    \item \textbf{Student}\\Student is one of the main Actors. The entire Platform is build with the aim to improve his/her coding skills, joining Battles, forming Teams, invinting other Students to join Battles, receiving Badges.
    \item \textbf{Educator}\\Educator is one of the main Actors. He/She manages Tournaments, alone or inviting other Educators, creating them and adding Battles. Educator creates Badges and can add extra point to Student's Score.
    \item \textbf{Battle}\\Within a Battle Students compete to improve their skills. It is the base entity on which is realized the entire Platform.
    \item \textbf{Team}\\This entity describes behaviors and interactions common to Students' Teams, that compete in a Battle, and a set of Educators, that manages a Tournament.
    \item \textbf{Badge}\\In order to gamify the Platform Badges can be created. These are assigned to a Student that achieves a specific goal.
    \item \textbf{Tournament}\\Tournaments are the context within Battles take place. They are managed by Educators and defines a specific topic for code developed for a hosted Battle.
    \item \textbf{Repository manager Platform}\\Repository management Platform is an external Platform that oversees code developed for a particular Battle.
    \item \textbf{Score}\\Score is the quantified evaluation of the uploaded code. It is automatically assigned by the Platform or can be defined by an Educator.
    \item \textbf{Application Programming Interface}\\ The Application Programming interface is a set of functions that denote the interface of a software system to other ones.
\end{enumerate}
\subsubsection[short]{Repository manager Platform specific definitions}
These definitions describe the technical jargon used while talking about repository management Platform functionalities.
\begin{enumerate}[label=$\bullet$]
    \item \textbf{Version Control System}\\A system that allows to save data in different versions keeping track of changes and authors.
    \item \textbf{Repository}\\A folder in which code is stored and managed by a version control system.
    \item \textbf{Action}\\Code that gets executed by the RMP on the repository once a certain condition is met (i.e. new upload, time of the day, etc...).
    \item \textbf{Commit}\\Act in which changes are saved in the VCS. Commits have details consisting in a short textual description known as Message, and author data consisting in RMP handle and e-mail address.
    \item \textbf{Push}\\Act in which commits are uploaded to the RMP Platform.
    \item \textbf{Fork}\\Act of duplicating repositories created by others in a personal one
\end{enumerate}
\subsection{Abbreviations and Acronyms}
\begin{enumerate}[label=$\bullet$]
    \item \textbf{CKB}\\CodeKataBattle, the name of the Platform.
    \item \textbf{RMP}\\Repository Manager Platform
    \item \textbf{VCS}\\Version Control System
    \item \textbf{API}\\Application Programming Interface
\end{enumerate}
\subsection{Reference documents}
\begin{enumerate}[label=$\bullet$]
    \item Requirement Analysis Specification Document (RASD)
\end{enumerate}
\subsection{Document Structure}
\begin{enumerate}
    \item \textbf{Introduction} This section provides a general description of the document purpose, structure and goals associated useful definitions, acronyms and abbreviations related to the software to develop. Moreover, is reminded to the reader the general description and purpose of the app for which is intended to provide instructions.
    \item \textbf{Architectural Design} Here are reported the main architectural components and qualities to implement in the software. A more detailed description of the app's functioning appears with respective modules and physical components that host them. In order to do this, the section is enriched with diagrams and specification to organize the code in the architecture making it work as intended.
    \item \textbf{User Interface Design} User Interface is crucial for the app's usability. In this section are provided principles to apply to realize the intended interface, providing mockups, examples and descriptions. It would be clarified how to connect interface components to the code.
    \item \textbf{Requirements Traceability} The purpose of the Document is to provide details to implement the software as intended. This means that is crucial to respect requirements and goals at all levels defined in the RASD. Each module and software components has to be matched with the requirement it brings to live. This procedure, reported in this section, is at the base of the project's coherence and it's essential for its correct realization.
    \item \textbf{Implementation, Integration and Test Plan} DD provides a more detailed analysis and description of what is needed to finally shape the project. In order to do that is vital to have a plan with the intention of achieving goals for the prefixed time without renouncing to quality. Proceeding step by step and following a logic based on software characteristics, is possible to make the app progressively grow meeting stakeholders, clients and developers needs, involving every actor in the process without compromising the code. For these reasons Implementation, Integration and Test plans are provided to build a coherent, robust, complete code.
\end{enumerate}