In this section is presented a mapping between the two documents, the Requirement Analysis and Specification Document and the Design Document. The mapping is done between the requirements and the components that implement them.\\
In the following table are reported the requirements and the components that implement them.
%\subsection{Component - Sequence Diagram - Requirement Mapping} \label{uc:mapping}
%\begin{center}
 %   \begin{tabu}{|X[.25]|X[.4]|X[.2]|} \hline \everyrow{\hline}
  %      Component & Sequence Diagram & Requirement \\
   %     Notification Manager & 2.5.2, 2.5.4, 2.5.5, 2.5.6, 2.5.7, 2.5.8, 2.5.9, 2.5.10, 2.5.11, 2.5.12  & R1, R7, R8, R14\\ 
    %    Account Manager & 2.5.3, 2.5.12  & R2, R17\\
        %Sign in Manager & 2.5.2 & R1\\
       % Log in Managerer & 2.5.3 & R2\\
      %  Search Manager & 2.5.15, 2.5.16& R4, R15\\
     %   Badge Manager & 2.5.12 & R17\\
    %    Tournament Manager & 2.5.4, 2.5.5, 2.5.6, 2.5.7, 2.5.8, 2.5.9, 2.5.10, 2.5.11, 2.5.12, 2.5.13, 2.5.14 & R3, R5, R6, R7, R8, R9, R10, R11, R13, R14\\
   %     Evaluation Manager & 2.5.12, 2.5.13 & R12\\
  %      Battle Manager & 2.5.13, 2.5.14 & R10, R12\\
 %       RMP Manager & 2.5.12, 2.5.13 & R12, R16\\
%    \end{tabu}
%\end{center}

\subsection{Component - Sequence Diagram - Requirement Mapping} \label{uc:mapping}
\begin{center}
    \begin{tabu}{|X[.3]|X[.4]|X[.3]|} \hline \everyrow{\hline}
        \nameref{cmp:cmp}& Sequence Diagram & Requirement \\
        Notification Manager & \ref{sq:2}, \ref{sq:4}, \ref{sq:5}, \ref{sq:6}, \ref{sq:7}, \ref{sq:8}, \ref{sq:9}, \ref{sq:11}, \ref{sq:12}  & R1, R7, R8, R14\\ 
        Account Manager & \ref{sq:3}, \ref{sq:12}  & R2, R17\\
        Sign in Manager & \ref{sq:2} & R1\\
        Log in Managerer & \ref{sq:3} & R2\\
        Search Manager & \ref{sq:15}, \ref{sq:16}& R4, R15\\
        Badge Manager & \ref{sq:12} & R17\\
        Tournament Manager & \ref{sq:4}, \ref{sq:5}, \ref{sq:6}, \ref{sq:7}, \ref{sq:8}, \ref{sq:9}, \ref{sq:11}, \ref{sq:12}, \ref{sq:13}, \ref{sq:14} & R3, R5, R6, R7, R8, R9, R10, R11, R13, R14\\
        Evaluation Manager & \ref{sq:12}, \ref{sq:13} & R12\\
        Battle Manager & \ref{sq:13}, \ref{sq:14} & R10, R12\\
        RMP Manager & \ref{sq:12}, \ref{sq:13} & R12, R16\\
    \end{tabu}
\end{center}

\subsection{Non-functional requirements' traceability}
The presented design of the platform allows to satisfy the non-functional requirements pointed in the RASD. In particular:
\subsubsection{Reliability and Availability and Maintainability}
One of the advantges of using a microservice architecture is that the platform has an intrinsically high level of reliablilty and availability. 
In fact, this kind of architecture allows for replication of services: this allows that issues to one instance of them can be promptly solved by another one serving the same feature. 
Moreover, although has been presented as all hosted on the same server, the application subsystem, structured in this way, can be easily distributed, allowing for a higher level of reliability and availability.
Finally, the use of a microservice architecture allows for a high level of maintainability, since the system is composed of many small services, that can be easily modified, updated or replaced without affecting the rest of the system.\\
\subsubsection{Security}
The Security of the system is guaranteed thanks the partitioning of the network in three subnetworks, where the once nearest to the internet are also the once that contain the less sensible data and application logic. In fact the use of the Demilitarized zone to host the Web Server, is the best solution to keep high both security and availability, since it masks the internal logic from the hostilities from the Internet.\\ 
Another possible point of attack could be the testing of user created code, since the testing environment, if badly implemented, could allow for the execution of malicious code bypassing all protection directly inside the application logic. 
Has already been described the use of containers for testing purposes, to avoid this.
\subsubsection{Portability}
Since the application is developed as a Web Application, it can be accessed via a Browser, which is a kind of software that is available for almost all kinds of computers capable of some software development. 
Moreover, this nature of the platform, allows for easy integration with smartphone and tablet systems, which, even if not capable of software development, can be used anyway for interfacing with the platform.
The use of the RMP allows for users not to be forced to use always the same computer for participating to tournaments, since they can access to their code and their standings from the internet.