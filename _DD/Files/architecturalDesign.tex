\subsection{Overview}
%CLIENT-SERVER
CKB is conceptualized from the ground basis as a Client-Server Platform. In fact, Users access services through a Web App, which constitutes the only virtual device able to receive commands to interact 
with CKB.
\useSvgWithCaption{./Images/overviewDiagrams/clientServerParadigm.svg}{0.35}{0.35}{Client-Server paradigm}
\\
The main logic is hosted in the server that manages all the functionalities of CKB, instead the client is only a graphic representation for the User.

Because the client, acting as a thin client, only displays the GUI to the User in order to expose the Platform's services and functionalities with their corresponding interfaces that it obtains by communicating with the Server through 
the Network, the client-server software organization that was chosen is Remote Representation.

The Web Browser hosting the Web App should ideally handle client communication. It must enable safe and efficient communication with the server via the HTTP and TCP protocols. 

To be used with all types of browsers, CKB User Interface code needs to be written. However, after their kind and privileges, servers would be assigned to protect Users' data, and the Platform's model would change in accordance 
with the guidelines.\\
\\
\subsection*{Microservices Architecture}
%MICROSERVICES
The fundamental capability of CKB is the ability to simultaneously manage and conduct multiple Battles, Tournaments, and receive multiple inputs from various clients, based on their type. 
The server must manage several requests at once, recognizing this through concurrency.

The Microservices architectural approach is one that can be selected. 
This option, which is supported by the availability of multiple servers, would allow each machine to host multiple service components that are properly replicated even in other machines. 
These components would each elaborate particular inputs pertaining to a particular topic or collection of related operations that make up a service.
Components can also be allocated according to what are the needs of the moment, if the number of requests increases, the number of components can be increased, and vice versa.

The concept would include DBMS services, provided thought proper APIs, as long as communication ones and the dispatcher itself, that would account all requests from Users redirecting them to the right Microservice. This architecture guarantees concurrency, modularity, witch positively affects maintenance, and Platform reactivity.

The method ensures modularity and improvements by enabling developers and authors to update CKB with new functionalities without altering the User interface in any way. 
It also allows the Platform to retain data and determine resilience while ensuring appropriate duplication of each component to prevent failures.\\
\\
%LAYER
CKB is thought to be implemented as a Three Layer Architecture:
\begin{enumerate}[label=$\bullet$]
    \item \textbf{Presentation Layer:} this layer's only purpose is to display to the User the application's product functionalities through a graphical interface.
    \item \textbf{Application Layer:} this layer is the central component of the Platform since it contains the entire application's logic and controls its functionalities.
    \item \textbf{Data Layer:} the application's whole data set is contained in this layer.
\end{enumerate}
\useSvgWithCaption{./Images/overviewDiagrams/3LayerArchitecture.svg}{0.65}{0.65}{Three Layer Architecture}
%TIER
CKB, once realized, would be a Four Tier Application:
\begin{enumerate}
    \item \textbf{Client:} The task assigned to this Tier is to gather input from users—both Educators and Students—and report their instructions into the system. It functions as a GUI.
    \item \textbf{Web Server:} It counts the parts assigned to create a secure connection between the Application Server and the Clients, which are the major actors. Web Server is outfitted with 
    suitable firewalls and a Demilitarized Zone (DMZ) to ensure the security and integrity of Application Server data.
    \item \textbf{Application Server:} It houses the elements that make up the Platform itself, the modules that carry out the primary needs and objectives outlined in RASD. This Tier can be 
    defined as the application Model, realized through a dispatcher-coordinated Microservices Architecture.
    \item \textbf{DBMS Server:} Data security, integrity, and preservation are the purview of DBMS. It will provide them with the relevant modules' requests according to their authorization for that particular data. 
\end{enumerate}
\useSvgWithCaption{./Images/overviewDiagrams/4TierArchitecture.svg}{0.8}{0.8}{Four Tier Application}
\newpage

\subsection{Component View}
\subsubsection{Microservices Architecture}
CKB's Application Layer is based on Microservices, that constitutes components in the system's architecture.
\useSvgWithCaption{./Images/componentDiagrams/microserviceArchitecture.svg}{0.88}{0.88}{Microservices Architecture}
\newpage
\subsubsection{Component Diagrams}
\useSvgWithCaption{./Images/componentDiagrams/componentDiagram.svg}{0.81}{0.81}{Component Diagram}
\newpage

\paragraph{Client}
The Web Browser module is the only module contained in the Client in this representation.
To establish a network connection between the Client and the CKB Server, a Web Browser is required. The Web Browser can also control the graphical elements that the Server sends to the client, which the User's GUI will employ.
\useSvgWithCaption{./Images/componentDiagrams/componentDiagramClient.svg}{0.3}{0.3}{Client Component Diagram}

\paragraph{Web Server}
The Web Browser's job is to route browser requests to the Application Server and get back responses from the Application Server. The Web Server just needs to route requests from and to the Network because you have to assume 
that it is located in the DMZ. It is composed by two modules:
\begin{enumerate}[label=$\bullet$]
    \item \textbf{Web Server Module} This module is in charge of receiving requests from the Client and forwarding them to the Application Server. It also receives responses from the Application Server and forwards them to the Client.
    \item \textbf{API Module} This module is in charge of providing the Web Server with the APIs that it needs to communicate with the Application Server. QUI VA SPIEGATO COME FUNZIONA L'API, LA RISPOSTA DELLE EMAIL
\end{enumerate}
\useSvgWithCaption{./Images/componentDiagrams/componentDiagramWebServer.svg}{0.5}{0.5}{Web Server Component Diagram}
\newpage

\paragraph{Application Server}
\begin{enumerate}
    \item \textbf{Dispatcher} The Dispatcher is the component that routes the requests to the right Microservice. It is invoked by the Web Server, which provides the request and the parameters. The Dispatcher will call the right Microservice, passing the parameters. The functionality of tthis component will be better shown in the dedicated section: \nameref{parr:dispatcher}.
    \item \textbf{Sign In Manager} It is the manager in charge of allow Users to sign in into the Platform through proper interface.
    \item \textbf{Log In Manager} This module is the access door to CKB's Platform. It memorizes all Users which are currently interacting with the system. The component exposes a 'logIn' Interface used by the User to log in 
    and by other components to verify if an existing User is, in fact, logged in. Finally, the interface allows logging out too.
    \item \textbf{RMP Manager} The manager allows the Platform to perform pull requests from corresponding repos, acting as a Client in face of RMP. It is used by Evaluation Manager which access 'pullRequests' Interface.
    \item \textbf{Notification Manager} This component manages particular notifications to be sent to Users, communicating with E-mail provider, which performs as a Server for this module. Requests are sent on other 
    components' indications, which are expressed through 'sendNotification' Interface.
    \item \textbf{Evaluation Manager} It is in charge of running test cases for every code file submitted by each Team. Basing on given parameters it would assign the corresponding Score.
    QUI VA SPIEGATO COME FUNZIONA L"ENVIROMENT DI ESECUZIONE
    \item \textbf{Badge Manager} It assigns Badges to Students who can be awarded.
    QUI VA SPIEGATO COME FUNZIONA L"ENVIROMENT DI ESECUZIONE DEI BADGE
    \item \textbf{Battle Manager} It allows managing Battles, updating the score of each Team.
    \item \textbf{Tournament Manager} It allows managing Tournaments, creating Teams, inviting Students and Educators, creating Battles and Badges, updating the Tournament score.
    \item \textbf{Account Manager} It is in charge of recruiting all information about a single Account, allowing the owner to update the Account with new personal data.
    \item \textbf{Search Manager} It performs all searches of Users and Tournaments within the Platform. It organizes the data properly from results got by DBMS and makes them available via its interface.
\end{enumerate}
VA SPIEGATO COME TUTTI QUESTI SI INTERFACCIANO CON IL DBMS
\newpage
\useSvgWithCaption{./Images/componentDiagrams/componentDiagramApplicationServer.svg}{0.73}{0.73}{Application Server Component Diagram}
\newpage
\subparagraph*{Dispatcher Component} \label{parr:dispatcher}
The dispatcher is the backbone of the whole microservice architecture, it manages load balancing, routing and communication between components and if necessay pawn or kill instances of components. 
Components are assumed to share interfaces between each other but before any kind of communication, need they need to call the dispatcher, who will give the component a valid reference to an instance of a component suitable to receive the request.
This allows the dispatcher to do load balancing, which it can decide which component make the request go to, and consequently which components replicate.
The DBMS is connected to the dispatcher also, and communications to it involve the dispatcher too. 
This is done to allow future scalability and flexibility, since the proposed solution involves the use of only one DBMS and one Database, but, if necessary, in future can be expanded without changing anything in the architecture.
Here follows the sequence diagram of what hereby described. It is to be intended as placed everywhere there is a communication between components in the following sequence diagrams.\\
\useSvgWithCaption{./Images/sequenceDiagrams/dispatcherSequenceDiagram.svg}{.70}{.70}{Dispatcher Sequence Diagram}
Where MicroserviceA and MicroserviceB are two generic components, Somewhere is a microservice, or Web Server or API manager or any component that triggers MicroserviceA to do somenthing in a procedure, and Dispatcher is the dispatcher component.
From this the dispatcher interface has also a method \hyperref[meth:dispGetData]{\textbf{getData()}} to get a reference to a component, necessary to allow components to communicate between each other.\\


\paragraph{DBMS Server}
The DBMS server's job is to keep all of the Platform's data safe and always accessible. Information about Users, Tournaments, and Battles are contained in the data.
\useSvgWithCaption{./Images/componentDiagrams/componentDiagramDBMSServer.svg}{0.4}{0.4}{Client Component Diagram}

\paragraph{External Services}
The role of the External Services is to provide the Platform with the functionalities that it needs to work properly. In particular, the E-mail Provider is used to send notifications to Users, while the RMP is used to perform pull requests.            
\useSvgWithCaption{./Images/componentDiagrams/componentDiagramExternalServices.svg}{0.5}{0.5}{External Services Component Diagram}
\newpage

\subsection{Deployment View}
\useSvgWithCaption{./Images/deploymentDiagram/deploymentDiagram.svg}{1.0}{1.0}{Deployment Diagram}

%TIER
As previously said, when CKB is deployed, it will include a Four Tiers Architecture that both reflects and extends upon the three Layers that rationally make up CKB.
In fact, additional components will be needed for the implementation in order to handle user and platform interactions as well as data module interaction.
\begin{enumerate}[label=$\bullet$]
    \item \textbf{Client:} Users input data into the Client, which displays responses from the Server and enables Users to navigate and interact with the Platform. In fact this Tier has its corresponding logical layer in the presentation one. 
    Since a thin client only has communication APIs with the server, it transmits the components needed to create the most recent one as well as the necessary data to display the GUI.
    The appropriate Student and Educator variants make up the User visualizing interface, representing, as specified before, the actual presentation layer; differences between them 
    are only at this architectural level unimportant to the operation of the system.
    \item \textbf{Web Server:} The primary entry point to the Platform is through the Web Server. It takes into account the interfaces that provide system access, the firewalls, the DMZ, and the module assigned to communicate with 
    clients. In particular input from Clients, once achieved the Server, enter into the DMZ and as first step pass thought a first firewall that analyzes them to verify the provenience of the same. If it is granted them access, Web Server registers Client that sent it, parsing the content of the request, evaluating in addition their correctness. 
    Furthermore, Web Server it is responsible for providing clients with information that updates the GUI and reflects changes made to the Platform. Web Server and Application Server are divided by a second firewall which role is to avoid undesired access to the actual model. Between Students, Educators, 
    and CKB, this Tier serves as a bridge. Web Server constitutes part of the implementation of the Application Layer.
    \item \textbf{Application Server:} The Platform, with its essential features, is the Application Server. It is constructed using a Microservices Architecture, which is made up of lone software modules that oversee a single 
    "service", or collection of tasks associated with a particular subject. Each module that performs a Microservice is better explained and analyzed in the previous and following section of this document. After parsing the requests, a dispatcher module that routes client requests to the appropriate service comes before the structure. Managing the interactions of several Users at once 
    will be possible with a proper distribution of the Application Server across Microservices. A firewall delimits this Tier with the DMZ as described above, a second one separates the App with Email-Provider and RMP and a third defines the boarder with DMBS.
    This last one avoids intrusions and damage into DBMS, while the fist two are deputed to protect Application Tier from external attacks or bad requests. Finally, this Server constitutes the reaming implementation of the Application Layer.
    \item \textbf{DBMS Server:} CKB keeps track of both User data and Platform operational data, such as Battle and Tournament records. DBMS is responsible for managing, securing, and storing the data. It permits access to 
    Microservices that need it, and it updates it as needed in compliance with regulations. This Tier is the actual implementation of logic Data Layer and it is protected by a Firewall that filters requests and operations from Application Server. 
\end{enumerate}

\begin{landscape}
\subsection{Runtime View}
\subsubsection{Sign in}
\useSvgWithCaption{./Images/sequenceDiagrams/signIn.svg}{1.5}{1.5}{Sign In Sequence Diagram}

\clearpage
\subsubsection{Log in}
\useSvgWithCaption{./Images/sequenceDiagrams/logIn.svg}{1.30}{1.30}{Log In Sequence Diagram}

\clearpage
\subsubsection{Join a Tournament}
\useSvgWithCaption{./Images/sequenceDiagrams/joinATournament.svg}{1.5}{1.5}{Join a Tournament Sequence Diagram}

\clearpage
\subsubsection{Invite Educator}
\useSvgWithCaption{./Images/sequenceDiagrams/inviteEducator.svg}{1.4}{1.4}{Invite an Educator Sequence Diagram}

\clearpage
\subsubsection{Invite Student}
\useSvgWithCaption{./Images/sequenceDiagrams/inviteStudent.svg}{1.5}{1.5}{Invite a Student Sequence Diagram}

\clearpage
\subsubsection{Receive Educator Invitation}
\useSvgWithCaption{./Images/sequenceDiagrams/receiveEducatorInvitation.svg}{1.5}{1.5}{Receive an Educator invitation Sequence Diagram}

\clearpage
\subsubsection{Receive Student Invitation}
\useSvgWithCaption{./Images/sequenceDiagrams/receiveStudentInvitation.svg}{1.5}{1.5}{Receive a Student Sequence Diagram}

\clearpage
\subsubsection{Create a Tournament}
\useSvgWithCaption{./Images/sequenceDiagrams/createATournament.svg}{1.5}{1.5}{Create a Tournament Sequence Diagram}

\clearpage
\subsubsection{Create a Battle}
\useSvgWithCaption{./Images/sequenceDiagrams/createABattle.svg}{1.5}{1.5}{Create a Battle Sequence Diagram}

\clearpage
\subsubsection{Join a Battle}
\useSvgWithCaption{./Images/sequenceDiagrams/joinABattle.svg}{1.4}{1.4}{Join a Battle Sequence Diagram}

\clearpage
\subsubsection{Create a Badge}
\useSvgWithCaption{./Images/sequenceDiagrams/createABadge.svg}{1.5}{1.5}{Create a Badge Sequence Diagram}

\clearpage
\subsubsection{Assign a Badge}
\useSvgWithCaption{./Images/sequenceDiagrams/assignABadge.svg}{1.5}{1.5}{Assign a Badge Sequence Diagram}

\clearpage
\subsubsection{Evaluate Code}
\useSvgWithCaption{./Images/sequenceDiagrams/evaluateCode.svg}{1.5}{1.5}{Evaluate Code Sequence Diagram}

\clearpage
\subsubsection{Educator Manual Evaluate Code}
\useSvgWithCaption{./Images/sequenceDiagrams/updateTournamentScoreEducator.svg}{1.5}{1.5}{Educator Manual Evaluate Code Sequence Diagram}

\clearpage
\subsubsection{User searches for Users}
\useSvgWithCaption{./Images/sequenceDiagrams/userSearchesForUsers.svg}{1.5}{1.5}{User searches for Users Sequence Diagram}

\clearpage
\subsubsection{Student searches for Tournaments}
\useSvgWithCaption{./Images/sequenceDiagrams/studentSearchesForTournament.svg}{1.5}{1.5}{Student searches for Tournaments Sequence Diagram}
%\restoregeometry
\end{landscape}
\clearpage

\subsection{Component Interfaces}
\useSvgWithCaption{./Images/componentInterface/componentInterface.svg}{1.0}{1.0}{Component Interfaces Diagram} 
For each component here are explained functions associated with respective interfaces.
%DA VERIFICARE SE INSERIRE ANCHE I GETTER
\begin{enumerate}
    \item \textbf{SignInManager Interface} 
            \begin{enumerate}[label=$\bullet$]
                \item \textbf{AccountManager signIn(String name, String surname, String email, String password, String rmpHandle)} This method allows User to register to the Platform providing name, email, password and a RMP handle. 
                The return is a Manager for that specific Account created.
            \end{enumerate}
    \item \textbf{LogInManager Interface}
        \begin{enumerate}[label=$\bullet$]
            \item \textbf{Boolean logIn(String email, String password)} When a user requests to enter CKB, whether they are registered, the manager is notified via this method. A Boolean value is returned as a result of the 
            invocation, informing the client whether the action was successful or unsuccessful based on the correctness of the account's existence parameters.
            \item \textbf{Boolean logOut(String email)} In order to log out form the system is needed as input the email of the User to discard from Users that are currently interacting with the Platform. A Boolean value communicates 
            the outcome.
            %\item \textbf{Boolean isLogged(String email)} Through this method, providing the email, the caller is able to know is the User who asks for the operation is actually logged in.
        \end{enumerate}
    \item \textbf{RMPManager Interface}
        \begin{enumerate}[label=$\bullet$]
            \item \textbf{Array<File> pullRequest(String repoLink)} In order to perform a Pull request to the right RMP repo by RMPManager, the component through the here described method, asks for the repo as a parameter. The return 
            value is the code pulled from the repo itself.
        \end{enumerate} 
    \item \textbf{NotificationManager Interface}
        \begin{enumerate}[label=$\bullet$]
            \item \textbf{Boolean sendNotification(String email, String text)} The only parameters required are the recipient email address and the content of the notification. The method reports a Boolean value to report the status 
            of the operation.
            \item \textbf{void receiveResponse(Int request, Boolean response, Array<String> info)} The method is invoked to provide the User's answer. The 'request' parameter is an integer that identifies the type of request 
            (eg: 0 <- join a Team, 1 <- add Educator to a Tournament), the 'response' parameter is a Boolean value that indicates whether the User has accepted or not and the 'info' Array contains the information about the response 
            and its content is different for each type of response (eg. "join to a Team": info[0] <- Student's email, info[1] <- Team, info[2] <- Tournament).
        \end{enumerate}
    \item \textbf{BattleManager Interface}
        \begin{enumerate}[label=$\bullet$]
            \item \textbf{void updateBattleScore(Int score, String Team)} The score of a Team is updated, via this method's invocation, providing the actual Score, which is an integer value. The Team's name is required too.
        \end{enumerate}
    \item \textbf{TournamentManager Interface}
        \begin{enumerate}[label=$\bullet$]
        \item \textbf{void createTeam(String myEmail, Array<String> emails, String Team)} The creation of the Team is deputed to this method. As input parameters are necessary the email of the User who invites others, list of emails 
        of other Students to invite, and the name of the Team. This method will interface with NotificationManager to send the invitation to the other Students. 
        \item \textbf{void joinStudentTournament(String email, String Team)} This method add a Student with the given email to the Team with the given name.
        \item \textbf{void joinEducatorTournament(String email)} This method add an Educator with the given email to the Tournament.
        \item \textbf{void addEducator(String email)} This method send an invitation to the Educator with the given email to join the Tournament. This method will interface with NotificationManager to send the invitation to the Educator.
        \item \textbf{void updateTournamentScore(String Team, Int score)} The Score of the Tournament has to be updated via this function, that requires the name of the Team and the integer value of the score.
        \item \textbf{Boolean addNewBattle(String name, String overview, String rmpRepo, Array<Int> evaluationParameters)} The method allows to add a new Battle to the current Tournament. It would be required the name of the Battle, 
        the RMP link from which Students will fork the repo and an array containing the evaluation parameter (eg. evaluationParameters[0] <- 25, evaluationParameters[1] <- 25, evaluationParameters[2] <- 25, evaluationParameters[3] <- 25,
        respectively for Functional, Timeliness, Quality and Manual).
        \item \textbf{void addNewBadge(String name, String description, File criteria, File photo)} The method allows to add a new Badge to the current Tournament. It would be required the name of the Badge, a description, the criteria 
        and a photo.
        \end{enumerate}
    \item \textbf{AccountManage Interface} 
        \begin{enumerate}[label=$\bullet$]
            \item \textbf{void updateAccount(String name, String surname, String email, String rmpHandle)} updateAccount is the function used by the User to update its personal information, providing new name, surname, email or repo 
            through which he/she would be identified in the system.
            \item \textbf{void assignBadge(BadgeManager badge)} Via this function is assigned a Badge to the Student, providing the corresponding BadgeManager.
        \end{enumerate}
    \item \textbf{SearchManager Interface} 
        \begin{enumerate}[label=$\bullet$]
            \item \textbf{Array<AccountManager> searchUser(String email, String name, String surname, String rmpHandle)} To search one or more Users SearchManager, thought the here presented method requires the email or name or surname
            of the User to search and returns the list of the Students'/Educators' AccountManagers of the ones who matched the criteria. Many overrides of the function will be present to perform the search based on just the available values.
            \item \textbf{Array<TournamentManager> searchTournament(String tournamentName, Array<String> properties)} To search one or more Tournaments SearchManager, thought the here presented method requires the name of the Tournament 
            to search and the properties of it (eg. properties[0] <- "Java", properties[1] <- "minimum 2 Students"). The return value is the list of the Tournaments' TournamentManagers of the ones who matched the criteria.
        \end{enumerate}
    \item \textbf{BadgeManager Interface}
        \begin{enumerate}[label=$\bullet$]
            \item \textbf{Boolean verifyCriteria(File test, String email)} BadgeManager is invoked thought this method to verify if some tests results meet criteria to assign a Badge to a specific User. If the answer is True, the Badge
            is assigned.
        \end{enumerate}
    \item \textbf{EvaluationManager Interface}
        \begin{enumerate}[label=$\bullet$]
            \item \textbf{Int autoEvaluation(File code, File Test, Array<Int> evaluationParameters)} The method allows to evaluate the code provided as input, basing on the evaluation parameters. Both the code and the test files are
            required. The return value is the score of the code.
        \end{enumerate}
    \item \textbf{Dispatcher Interface}
        \begin{enumerate}[label=$\bullet$]
            \item \textbf{void dispatch(String request, Array<String> parameters)} The Dispatcher is the component that routes the requests to the right Microservice. It is invoked by the Web Server, which provides the request and the 
            parameters. The Dispatcher will call the right Microservice, passing the parameters.
            \item \textbf{MicroserviceData getData(String microserviceName)} \label{meth:dispGetData}The method is invoked by a microservice to know the data of an other microservice as discussed above \nameref{parr:dispatcher}. The parameter is the name of the microservice that is requesting the data.
        \end{enumerate}
\end{enumerate}

\subsection{Selected architectural styles and patterns}
\subsubsection{Four-Tier Architecture}
Flexibility, scalability and load distribution.
\subsubsection{Model View Controller}
SCRITTO CON COPILOT XD
The MVC pattern is used to separate the data model, the presentation layer and the logic layer. This pattern is used to separate the data model, the presentation layer and the logic layer. The MVC pattern is used to separate the data model, the presentation layer and the logic layer.
\subsection{Other design Decisions}
\subsubsection{Relational Database}
QUI VOGLIAMO METTERE ER?
\subsubsection{API}
\subsubsection{Scale-Out}
SCRITTO CON COPILOT XD
With Microservices Architecture, the Platform is able to scale-out, in order to manage multiple requests at the same time. In fact, each Microservice is able to run on different machines, and the Platform is able to replicate them in order to avoid failures.