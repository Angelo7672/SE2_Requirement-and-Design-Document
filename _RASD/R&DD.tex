\documentclass{article}
\usepackage{graphicx} % Required for inserting images
\usepackage{enumitem}
\usepackage{tabularx}
\usepackage[table]{xcolor}


\title{R\&DD}
\author{Angelo Attivissimo, Isaia Belardinelli, Carlo Chiodaroli}
\date{A.A. 2023 - 2024}

\begin{document}

\maketitle

\section{Introduction}
CodeKataBattle is a platform designed to helps students improve their software development skills by training and compete in tournaments against each other.
The student are supervised by Educators during the whole experience in CodeKataBattle: educators in fact are the supervisor of the platform, 
they create the code battle and the tournament.
\subsection{Purpose}
The aim of the platform is to give a practical teaching to the students: they can improve their software development skills using their knowledge about one or
more programming languages to resolve puzzles about programming or developing an entire product from scratch. In particular they will in competition one with the 
others following the schema of the "kata battle". This model, used by martial arts athletes, consists of repeating the exersice to constantly improve the result. 
Students are encouraged to collaborate with each other in a way to improve also the collaboration skills. Finally their work is evaluated by automated processes 
within the platform or by educators.
\subsubsection{Goal}
\textbf{Platform Goals}
\begin{enumerate}[label=$\bullet$ \textbf{GP\arabic*:}]
    \item \textbf{Allow Users to subscribe to the platform}\\ciao
    \item \textbf{Allow Users to search for other users}
    \item \textbf{Allow Educators to create tournaments}
    \item \textbf{Allow Educators to invite other educators to manage tournaments}
    \item \textbf{Allow Educators to create battles}
    \item \textbf{Allow Educators to define badges}
    \item \textbf{Allow Educators to score students}
    \item \textbf{Allow Students to get notificated about new tournaments}
    \item \textbf{Allow Students to subscribe to the tournament}
    \item \textbf{Allow Students to invite other Students to partecipate to a tournament}
    \item \textbf{Allow Educators to define badges}
\end{enumerate}
\textbf{User Goals}
    \begin{enumerate}[label=$\bullet$ \textbf{GU\arabic*:}]
        \item \textbf{Allow Users to subscribe to the platform}
        \item \textbf{Allow user to do the login}
        \item \textbf{Allow users to search torunaments and user}
    \end{enumerate}
\textbf{Student Goals}
\begin{enumerate}[label=$\bullet$ \textbf{GS\arabic*:}]
    \item \textbf{allow students to compete each other}
    \item \textbf{allow students to be rewarded for special achievement}
    \item \textbf{allow student to forming and working in team}
    \item \textbf{allow student to have works evaluated}
\end{enumerate}
\textbf{Educator Goals}
\begin{enumerate}[label=$\bullet$ \textbf{GE\arabic*:}]
    \item \textbf{allow educator to create tournament}
    \item \textbf{allow educator to create code kata battle}
    \item \textbf{allow educator to create badge}
    \item \textbf{allow educator to evaluate students}
\end{enumerate}
\section{Scope}
An user should register to the platform as student or educator. The student see in his homepage the tournaments, his profile, and in other he has the possibility to search other users registered in the platform.
The torunaments shown in the home page are described with a title and the programming language. In also the student can search other torunament in the platform. Clicking on a torunament, the student can see other details on that tournament: the deadline of inssubscription and
the duration of the tournament, the minum and maximum number of student in a team for that torunament, the number of the battle composing the torunament. When he subscribe to a tournament, he has to soddisf the minum nuber pf the team so if he is the first component of the team subscribing to that 
tournament he has to invite other student to the torunament if the minum number of student it is greater than one; the other student invited by the first one will register to the tournament by the invite.
\\
\\
The platform should permit to Student and Educators to partecipate to tournaments of Code kata battle, receiving battle and tournament score and badges for the first category of user, and create tournament and battle, create badge for the second category of user.
After the registration, the platform allow the subscription of a student to a tournament and to invite other students. The platform has to evaluate the code written by the students by automatic revision such as test passed, time of execution, security and realiability, and has to assign badges to the sutdent that have all the requrements of a badge suddisfied.
The paltform should permit to an Educator, after it registration to the platform, to create torunamens, and if i is necessary, adding other educators to that tournament to be helped in the creation of the battles or in the manual evaluation of the code writtten by the students. The educator in others has the possibility to create badge with one or more rules that are 
requirements that the students have to satisfied to receive that badge.


\begin{center}
    \begin{table}
        \rowcolors{3}{black!15}{white}
        \begin{tabularx}{\textwidth}{| c| c| c|}
            \hline
            \rowcolor{blue!50}
            Phenomenon                                 & Who controls it? & Is shared? \\
            \hline
            User wants to improve his softwre developing skill                    & W                & N          \\
            Educator wants to create a code kata battle                        & W                & N          \\
            Educator wants to create a tournament                        & M                & Y          \\
            Student forks the directory on GitHub                                 & W                & Y          \\
            Student creates a workflow Action on GitHub               & M                & N          \\
            User retrieves a new number                & W                & Y          \\
            Generate a unique number                   & M                & N          \\
            Visualize queue number                     & W                & Y          \\
            Check time-out                             & M                & N          \\
            Notify number is going to be called        & M                & Y          \\
            Generate a QR code                         & M                & N          \\
            Visualize QR code                          & M                & Y          \\
            Attendant scans QR code                    & W                & Y          \\
            Check QR code validity                     & M                & N          \\
            Customer retrieves a number on spot        & W                & Y          \\
            User selects day/time of slot              & W                & Y          \\
            Check availability of slot                 & M                & N          \\
            User adds visit duration                   & W                & Y          \\
            Check visit duration validity              & M                & N          \\
            User indicates list of items               & W                & Y          \\
            Confirm slot has been reserved             & M                & Y          \\
            Cross data                                 & M                & N          \\
            Make recommendation of alternative slots   & M                & N          \\
            Visualize suggestions of alternative slots & M                & Y          \\
            Check customers balance                    & M                & N          \\
            Accept receipt of notifications            & W                & Y          \\
            Search periodically for available slots    & M                & N          \\
            Notify for available slots                 & M                & Y          \\

            \hline
        \end{tabularx}
        \caption{phenomena table}
    \end{table}
\end{center}
\end{document}