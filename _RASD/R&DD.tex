\documentclass{article}
\usepackage{graphicx} % Required for inserting images
\usepackage{enumitem}
\usepackage{tabularx}
\usepackage[table]{xcolor}


\title{R\&DD}
\author{Angelo Attivissimo, Isaia Belardinelli, Carlo Chiodaroli}
\date{A.A. 2023 - 2024}

\begin{document}

\maketitle

\section{Introduction}
CodeKataBattle is a platform designed to helps students improve their software development skills by training and compete in tournaments against each other.
The student are supervised by Educators during the whole experience in CodeKataBattle: educators in fact are the supervisor of the platform, 
they create the code battle and the tournament.
\subsection{Purpose}
The aim of the platform is to give a practical teaching to the students: they can improve their software development skills using their knowledge about one or
more programming languages to resolve puzzles about programming or developing an entire product from scratch. In particular they will in competition one with the 
others following the schema of the "kata battle". This model, used by martial arts athletes, consists of repeating the exersice to constantly improve the result. 
Students are encouraged to collaborate with each other in a way to improve also the collaboration skills. Finally their work is evaluated by automated processes 
within the platform or by educators.
\subsubsection{Goals}
\textbf{Platform Goals}
\begin{enumerate}[label=$\bullet$ \textbf{GP\arabic*:}]
    \item \textbf{Allow Users to subscribe to the platform}\\In order to interact with the platform and its Users, Students and Educators have to be allowed to register. The satisfaction of this goal will concent them to be identified.
    \item \textbf{Allow Educators to create tournaments}\\Battles are fought within a Tournament context, which is created by an Educator. So it is a goal of the platform to allow them in this. 
    \item \textbf{Allow Educators to invite other educators to manage tournaments}\\Allowing Educators to invite other Educators will empower them to create Battles within the Tournament context. 
    \item \textbf{Allow Educators to create battles}\\Both Educator that created the Tournament and the ones invited should be allowed to create Battles. 
    \item \textbf{Allow Educators to define badges}\\Badges are created to gamify the platform and encourage Students to do their best. Educators are allowed to create Badges using different criterias.
    \item \textbf{Allow Educators to score students}\\In order to make Students conscient about their work Educators are allowed to assign scores on top of the one defined by the platform itself.
    \item \textbf{Allow Students to get notificated about new tournaments}\\To concent Student to participate, they should be notified of new Tornaments created.
    \item \textbf{Allow Students to subscribe to the tournament}\\To participate to the platform and being scored Students have to fight in "kata battles", which are defined within Tournament context.
    \item \textbf{Allow Students to invite other Students to partecipate to a tournament}\\The purpose of the platform is to improve Students' coding skills. This is done even throught teams that Students have to be allowed to autonomously form. To do so, they should be able to search for other Students within the platform.
\end{enumerate}
\textbf{User Goals}
    \begin{enumerate}[label=$\bullet$ \textbf{GU\arabic*:}]
        \item \textbf{Allow Users to do the login}\\Registered Users to be identified should be allowed to login in the platform.
        \item \textbf{Allow Users to search tournaments}\\Both Educators, to ask for access to Tournament in order to create Battles, and Students, to join them, Users should be allowed to search for Tournaments.
        \item \textbf{Allow Users to search for other users}\\To collaborate with other Users, they have to be allowed to search for others, both Educators, to share Tournaments accesses and create Battles within it, and Students, to invite others to a Battle.
    \end{enumerate}
\textbf{Student Goals}
\begin{enumerate}[label=$\bullet$ \textbf{GS\arabic*:}]
    \item \textbf{Allow Students to compete each other}\\The principle behind the platform is to improve Students' skills, following "Kata Battle" mode. This implies Students to compete with each other on their own or in teams.
    \item \textbf{Allow Students to be rewarded for special achievement}\\Students should be allowed to receive scores for their work. In addition to this special achievements are rewarded, according to criteria decided by Educators.
    \item \textbf{Allow Students to form and work in team}\\To compete with each other Students can form teams. Working together they improve their collaboration skills.
    \item \textbf{Allow Students to have works evaluated}\\To be conscious of the quality of their work, Students have to be allowed to be have their work evaluated.
\end{enumerate}
\textbf{Educator Goals}
\begin{enumerate}[label=$\bullet$ \textbf{GE\arabic*:}]
    \item \textbf{Allow Educators to create Tournament}\\Battles are created within a Tournament context, which Educators are allowed to create.
    \item \textbf{Allow Educators to create Code Kata Battle}\\Educators, both the one that created the Tournament and the ones who join it, should be allowed to create Battles, within Students will compete.
    \item \textbf{Allow Educators to create Badges}\\To reward Students for special achievement, Educators should be allowed to create Badges.
    \item \textbf{Allow Educators to evaluate Students}\\Scores, on top of the ones computed by the platform itself, can be assigned by the Educator.
\end{enumerate}
\section{Scope}
An user should register to the platform as student or educator. The student see in his homepage the tournaments, his profile, and in other he has the possibility to search other users registered in the platform.
The torunaments shown in the home page are described with a title and the programming language. In also the student can search other torunament in the platform. Clicking on a torunament, the student can see other details on that tournament: the deadline of inssubscription and
the duration of the tournament, the minum and maximum number of student in a team for that torunament, the number of the battle composing the torunament. When he subscribe to a tournament, he has to soddisf the minum nuber pf the team so if he is the first component of the team subscribing to that 
tournament he has to invite other student to the torunament if the minum number of student it is greater than one; the other student invited by the first one will register to the tournament by the invite.
\\
\\
The platform should permit to Student and Educators to partecipate to tournaments of Code kata battle, receiving battle and tournament score and badges for the first category of user, and create tournament and battle, create badge for the second category of user.
After the registration, the platform allow the subscription of a student to a tournament and to invite other students. The platform has to evaluate the code written by the students by automatic revision such as test passed, time of execution, security and realiability, and has to assign badges to the sutdent that have all the requrements of a badge suddisfied.
The paltform should permit to an Educator, after it registration to the platform, to create torunamens, and if i is necessary, adding other educators to that tournament to be helped in the creation of the battles or in the manual evaluation of the code writtten by the students. The educator in others has the possibility to create badge with one or more rules that are 
requirements that the students have to satisfied to receive that badge.
\subsection{Phenomena}
\begin{center}
    \begin{table}
        \rowcolors{3}{black!15}{white}
        \begin{tabularx}{\textwidth}{| c| c| c|}
            \hline
            \rowcolor{blue!50}
            Phenomenon                                                            & Who controls it? & Is shared? \\
            \hline
            User wants to improve his softwre developing skill                    & W                & N          \\
            Educator wants to create a code kata battle                           & W                & N          \\
            Educator wants to create a tournament                                 & W                & N          \\
            Student forks the directory on GitHub                                 & W                & N          \\
            Student creates a workflow Action on GitHub                           & W                & N          \\
            Student push the file on GitHub repository                            & W                & N          \\
            Student write code                                                    & W                & N          \\
            User registers to CodeKataBattle                                      & W                & Y          \\
            User logins in CodeKataBattle                                         & W                & Y          \\
            User searchs for tournaments                                          & W                & Y          \\
            User searchs for users                                                & W                & Y          \\
            GitHub Action workflow notifies CodeKataBattle                        & W                & Y          \\
            Educator creates tournament                                           & W                & Y          \\
            Educator assigns optional scores to the students                      & W                & Y          \\
            Educator grant access to other educators                              & W                & Y          \\
            Educator creates battles                                              & W                & Y          \\
            Educator create a badge                                               & W                & Y          \\
            Student invites other student in a team                               & W                & Y          \\
            Student subscribe to a tournament                                     & W                & Y          \\
            Student gets notified of new tournament                               & M                & N          \\
            Student gets notified of being invited to partecipate to team         & M                & N          \\
            Educator gets notified of being invited to manage a tournament        & M                & N          \\
            Student gets score                                                    & M                & N          \\
            Student gets badge                                                    & M                & N          \\
            Educator gets notified about student push                             & M                & N          \\

            \hline
        \end{tabularx}
        \caption{phenomena table}
    \end{table}
\end{center}
\subsection{Definitions}
\begin{enumerate}[label=$\bullet$]
    \item \textbf{Platform}\\The all "Code Kata Battle" platform. This entity is inserted in order to rappresent system's behaviours in different situations, without explaining in details its implementation or workflow.
    \item \textbf{User}\\It is the generic User, Educators or Students depending on specific situation. It would be prefered to other definitions to model common behaviours and scenarios within the other two entities.
    \item \textbf{Student}\\Student is one of the main actors. His/Her scope is to improve its coding skills using the Platform. It will interact with it in a proper way in order to have its work evaluated, to collaborate with other Students and to join Battles.
    \item \textbf{Educator}\\Educator is the other main actor of the Platform. His/Her scope is to create Tournaments and Battles in which Students can compete
    \item \textbf{Battle}\\With this entity is described the single challenge Students have to face. It is the center around which all the platform is build, and only throught it would be achieve by Students the goal to improve their skills.
    \item \textbf{Group}\\The entity describes a set of Students, which are working in team on a single Battle, or of Educators, which creates Battles in a Tournament.
    \item \textbf{Badge}\\With Badges Students are rewarded for special achievements. This entity is introduced to gamify the platform.
    \item \textbf{Tournament}\\The Tournament is the context within Battles are created. A Tournament can hosts multiple Battles and rapresents the topic in which they take place.
    \item \textbf{GitHub}\\GitHub's entity models the omonimous repository manager's interactions with "Code Kata Battle".
    \item \textbf{Score}\\Throught Score the platform and/or an Educators evaluates Student's works. Each Student has its own Score history assigned, visible to other Students and Educators.
\end{enumerate}
\end{document}