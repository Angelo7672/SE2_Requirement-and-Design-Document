Computer Science is a field where theory and practice are thightly coupuled and both very important.
Infact in theory it can be easily studied by reading books or following courses on specific topics, but in reality being able to practice the subject is the best way to consolidate studies from pure erudition to knowledge.
Moreover coworking with other people becomes a moment of imporving personal knowledge with other people experience.
But for some people be able to work with other people is difficult for a multitude of reasons i.e work, geography ect... 
To help these people the CodeKataBattle platform aims to be a place where students can work in teams monitored by educators competing against other students programming code that gets scored and can be iteratively imporved, in order to reach the best score and win the tournament.
\subsection{Product Perspective}
\subsubsection{Scenarios}
\begin{enumerate}[label=$\bullet$ \textbf{SC\arabic*:}]
    \item \textbf{Student signs in to the platform to participate to tournaments}\\ Alice is studying programming all by herself and wants to evaluate her work. She accesses the Platform website and is intreagued to try it out. Alice signs in as a student giving her details like name, surname, e-mail,  year of birth and RMP handle. Once done that she will receive updates on new open tournaments. Moreover she can log in and be presented with a list of tournaments to which she can subscribe.
    \item \textbf{Student wants to participate to a tournament}\\ Dan, after studying a bit of javascript wants to apply its studies in a project. In order to do so he wants to participate in a tournament. he is already subscribed to the platform, so he can just log in to it and search a suitable tournament. After being presented of some valid choices he clicks on one of them to see more details like description, educators, subscription deadline, subscribed teams and available badges. If he likes it, he clicks a "Subscribe" button, else goes back and continues to search.
    \item \textbf{Student wants to participate to tournament with friend}\\ Bob, a user of the platform, got notified of the creation of an intrasting tournament in which he wants to partecipate with his friend Carlo. Bob logs in to the website and subscribes to the tournament sending an invite to Carlo. Carlo gets notified of the invite and, once logged in to the platform, can choose to accept or to deny the invite; if he accepts Bob and Carlo result subscribed to the tournament as a team.
    \item \textbf{Educator signs in to the platform to create tournaments}\\ Eve is a teacher that, in order to widen her didactic offer, wants to create some tournaments for her students. She accesses the Platform website and is intreagued to use it. Eve signs in as a educator giving her details like name, surname, e-mail, year of birth, istitution and RMP handle. Once done that she logs in and is presented with a list of managed tournaments, empty for her, and the possibility to create new ones.
    \item \textbf{Educator wants to create a tournament}\\ Faith is a user of the platform signed in as an Educator and wants to create a tournament. She logs in to the website and is presented with the homepage, in which she can start to create the tournament by clicking a specific button. After clicking it she can enter details like Tournament name and detail, subscription deadline and badges, then click a "Create tournament" button that will create the tournament and bring her to the manager tournement page. 
    \item \textbf{Educator wants to manage a tournament with a colleague}\\ Grace is an educator who is managing a tournament, she finds that the work behind creating all battles of the tournament is a bit too dawnting, so she wants help. To do so she goes to the tournament managment page and by clicking a "add Educator" button she can send an invite to manage the tournament to her friend Heidi, an other educator registered to the platform. Heidi can accept the invite and create new battles for the tournament.
    \item \textbf{Team wants to work on a battle}\\ A team has subscribed to a tournament, and after the start of the tournament wants to work on a battle. Once on the tournament page, Isaac, a student of the team, is presented with the list of the available battles, he chooses one and gets to an other page with name, details, team scoreboard and a link to the repository in the RMP. Isaac is already registered in the RMP and can fork the repository, creating an effective copy of it in its profile. At this point he has to invite all other team members to work in it by checking their RMP handles in their profiles. All team members have to accept the invite, and can start to work on it. Te owner of the forked repository has to put the link of it in the battle details to signal the platform of it's existence.
    \item \textbf{Team wants to have work evaluated}\\ A team has worked on the battle and wants to check how is the quality of his work. In order to do so they only need to commit their changes and push their work to the RMP, this will be able to alert the platform of the new changes and will set off a pull of the files to be evaluated. After running all tests, these will generate points that will be combined to form a score. this score will be published near the name of the team in the leaderboard of the battle, and will update also the main tournament score of that team. Possible checks for badge assingment can be also done, and if satisfied, badges can be granted.
    \item \textbf{Team wants to see how it is placed in the tournament}\\ Justin, a student participating in a tournament in a team, wants to know how it is going. To do so he can log in to the platform and go to the page of the tournament, on which he can see the leaderboard where he can find the overall score of the team. Justin can also check specific battles going to their specific page where a specific battle leaderboard is shown.
    \item \textbf{User wants to check out other user's profile}\\ Michael, a user of the platform, wants to find the profile of his friend Nick to checkout on him. He logs in to the website in which he can search for other users via a specic bar, there searches for Nick, and clicking on his name goes to the profile page. Here he can see his details like: user type, name, surname, RMP handle and badges.
\end{enumerate}
\subsection{Domain Model}
\subsubsection{Class diagram}
\usesvg{./Images/UML/entityClassDiagram.svg}
\subsubsection{Repository Manager Platform}
The Repository Manager Platform keeps the code of all battles of all tournaments of all teams. This can be a first or third party system that has to have some characteristic to be suitable to work on the CKB platform.\\
The repository is a folder in which code is stored and managed by a verstion control system (like Git) in which the code is saved via commits, special "save" operation of multiple files that are asssociated with comments and can be reverted as needed.\\
The repository manager platform is an online platform on which users can upload code for sharing purposes, this can be GitHub or GitLab (the two most famous), and has to have two main functionalities:\\
    an API (Application Programming Interface), techincally needed for the CKB platform to pull (download, get) code to be analized and scored.\\
    the possibility of running automatic actions (like GitHub's Workflows) that trigger once new changes are pushed to the RMP platform, needed to signal the platform to run tests on the code.
\subsubsection{Sequence diagram}