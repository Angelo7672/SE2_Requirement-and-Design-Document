\subsection{External Interface Requirements}

\subsubsection{User Interfaces}
In order to access the platform the crucial interface needed by the User is the one provided by a web browser. In fact, it is sufficient to reach the CKB's URL to start with log in or sign up operations, described in the scenarios, or, 
after authentication, interact with the platform. The platform software, in fact, is a Web App. Instead, to join a Battle a RMP link would be required.

\subsubsection{Hardware Interfaces}
Users have to provide themselves with a device able to access internet. It would be sufficient that it is equipped with a Wi-Fi and/or Ethernet interface. Of course would be crucial that it provides adequate components to allow Users 
to interact with the platform, showing its interfaces.

\subsubsection{Software Interfaces}
As defined above, a web browser is the only software needed to access the platform. The interfaces that have to be supported are the ones defined by the Web Page rendering. About RMP, the platform software run on the server would be 
equipped with appropriate interfaces to interact with the RMP provided by the Student or the Team in the steps described in previous sections of this document.
\subsubsection{Communication Interfaces}
Communication Interfaces needed are the one necessary to access the internet. So for the User, TCP and HTTP interfaces would be crucial to reach the server on which the platform runs, while for the CKB's app it is fundamental to 
interact with RMP.

\newpage

\subsection{Functional Requirements}
Here follows a list of the platform functional requirements:
\begin{enumerate}[label= \textbf{R\arabic*}]
    \item The platform allows Users to sign in to the platform itself either as Student either Educator. \label{req:reqSignin}
    \item The platform allows Users to log in to the platform. \label{req:reqLogin}
    \item The platform shows Students the list of available Tournaments. \label{req:reqShowTournaments}
    \item The platform allows registered Students to search for Tournaments. \label{req:reqSearchForTournament}
    \item The platform allows registered Students to subscribe to Tournament. \label{req:reqTournamentSubscription}
    % \item The platform allows registered Students to join a Battle. \label{req:reqJoinBattle}
    \item The platform allows registered Students to participate to Tournaments as a Teams. \label{req:reqCreateTeam}
    \item The platform allows registered Students to invite other Students to join a Team. \label{req:reqJoinTeam}
    \item The platform allows registered Educators to create Tournaments. \label{req:reqCreateTournaments}
    \item The platform allows registered Educators to create Battles in the context of a Tournament. \label{req:reqCreateBattle}
    \item The platform allows registered Educators to create Badges in the context of a Tournament. \label{req:reqCreateBadge}
    \item The platform assigns a Battle Score to Teams' work. \label{req:reqEvaluateCode}
    \item The platform provides a Teams' ranking based on the Tournament Score within a Tournament context. \label{req:reqRankingsUpdate}
    \item The platform allows registered Educators to add other Educators to a Tournament. \label{req:reqJoinManagement}
    \item The platform allows Users to search for other Users. \label{req:reqSearchForUsers}
    \item The platform interacts properly with different RMPs to acquire the latest versions of code uploaded by related Teams. \label{req:reqPullRMP}
    %\item The platform should be able to manage multiple requests at the same time \label{req:req13}?
    \item The platform awards Badges to deserving Students \label{req:reqAssignBadge}
\end{enumerate}

\newpage

\subsubsection{Use case diagrams}
\useSvgWithCaption{./Images/UML/useCaseDiagram/useCase1.svg}{0.95}{0.95}{Use case diagram of the login and the invitation}
\useSvgWithCaption{./Images/UML/useCaseDiagram/useCase2.svg}{0.93}{0.93}{Use case diagram of the Battle and the Tournament}

\newpage

\subsubsection{Use cases and associated sequence diagrams}
Here follow Use Case tables followed by respective sequence diagrams.
\paragraph{UC1 - User signs in to the Platform} \label{uc:uc1} 
\begin{center}
    \begin{tabu}{|X[.2]|X|} \hline \everyrow{\hline} 
        Name & User signs in to the Platform \\
        Actors & User \\ 
        Entry Condition & User has a valid e-mail address and valid RMP handle\\ 
        Event Flow & \begin{tabu}{X X[50]}
            1& At Homepage click on "Sign in" button\\
            2& System shows User the registration form\\
            3& User fills form with data, caring to choose his role accordingly\\
            4& User clicks on "Submit" button \\
            5& Platform saves submitted information\\
            6& Platform sends e-mail confirmation link to User\\
            7& User confirms e-mail by clicking confirmation e-mail\\
        \end{tabu} \\
        Exit Condition & User correctly registered in the Platform, User needs to login to use the Platform\\
        Exception & User provides an Email already in use.\\
        \specialReqLabel & - \\ 
    \end{tabu}
    \tableEntryByLabel{uc:uc1}
\end{center} 
"\nameref*{uc:uc1}" is a generalization of:\\
"\nameref{uc:uc1a}" and \\ "\nameref{uc:uc1b}".
\clearpage
\paragraph*{UC1a - Student signs in to the Platform} \label{uc:uc1a} 
\begin{center}
    \begin{tabu}{|X[.2]|X|} \hline \everyrow{\hline} 
        Name & Student signs in to the Platform \\ 
        Actors & Student \\ 
        Entry Condition & Same as \nameref{uc:uc1} \\ 
        Event Flow & Same as \nameref{uc:uc1} with exception of \newline \begin{tabu}{X X[50]}
            3& Student chooses Student role while filling in form\\
        \end{tabu} \\
        \exitCondLabel & Same as \nameref{uc:uc1}\\
        Exception & Same as \nameref{uc:uc1}\\
        \specialReqLabel & Same as \nameref{uc:uc1}\\ 
    \end{tabu}
    \tableEntryByLabel{uc:uc1a}
\end{center}
\useSvgWithCaption{./Images/UML/sequenceDiagram/sequenceDiagramUC1a.svg}{1.0}{1.0}{UC1, UC1a, Sequence Diagram - Student signs in to the Platform}
\clearpage

\paragraph*{UC1b - Educator signs in to the Platform} \label{uc:uc1b} 
\begin{center}
    \begin{tabu}{|X[.2]|X|} \hline \everyrow{\hline}
        Name & Educator signs in to the Platform \\ 
        Actors & Educator \\ 
        Entry Condition & Same as \nameref{uc:uc1} \\ 
        Event Flow & Same as \nameref{uc:uc1} with exception of \newline \begin{tabu}{X X[50]}
            3& Educator chooses Educator role while filling in form\\
        \end{tabu} \\
        Exit Condition & Same as \nameref{uc:uc1}\\
        Exception & Same as \nameref{uc:uc1}\\
        \specialReqLabel & Same as \nameref{uc:uc1}\\ 
    \end{tabu}
    \tableEntryByLabel{uc:uc1b}
\end{center}

\useSvgWithCaption{./Images/UML/sequenceDiagram/sequenceDiagramUC1b.svg}{1.0}{1.0}{UC1, UC1b Sequence Diagram - Educator signs in to the Platform}
\clearpage
\paragraph*{UC2}
\begin{center}
    \begin{tabu}{|X[.2]|X|} \hline \everyrow{\hline}
        Name & Student subscribes to tournament\\ 
        Actors & Student \\ 
        Entry Condition & - \\ 
        Event Flow & \begin{tabu}{X X[50]}
            1& Student logs in to the platform\\
            2& Student is in it's home page and is presented with a list of available tournaments\\
            3& Student scrolls through the list of available tournaments\\
            4& Student can ask to see more tournaments via the "See more" button\\
            5& Student clicks on tournament entry to see tournament details\\
            6& Student clicks again on tournament if not interested to hide details\\
            7& Student clicks on "subscribe" button to subscribe to tournament\\
            8& Platform registers tournament subscription\\
            9& Platform creates new team for that student\\
            10& Platform sends confirmation e-mail to student\\
            11& Platform gives visual feedback on web page of effective subscription\\
        \end{tabu} \\
        Exit Condition & Student is subscribed to tournament and knowing it\\
        Exception & Subscription deadline is passed, student cannot subscribe to tournament\\
        Special \newline Requirement & - \\ 
    \end{tabu}
\end{center}
\useSvgWithCaption{Images/UML/sequenceDiagram/sequenceDiagramStudentSubscribesToTournament.svg}{1.0}{1.0}{UC2 Sequence Diagram - Student subscribes to the tournament}
\clearpage
\paragraph*{UC3 - Student subscribes to Tournament} \label{uc:uc3}
\begin{center}
    \begin{tabu}{|X[.2]|X|} \hline \everyrow{\hline}
        Name & Student subscribes to Tournament\\ 
        Actors & Student \\ 
        Entry Condition & Tournament is in "ONLINE" state\\ 
        Event Flow & \begin{tabu}{X X[50]}
            1& Student logs in to the Platform\\
            2& Student is in its home page and is presented with a list of available Tournaments\\
            3& Student scrolls through the list of available Tournaments\\
            4& Student can ask to see more Tournaments via the "See more" button\\
            5& Student clicks on Tournament entry to see Tournament details\\
            6& Student clicks again on Tournament if not interested to hide details\\
            7& Student clicks on "subscribe" button to subscribe to Tournament\\
            8& Platform registers Tournament subscription\\
            9& Platform creates new Team for that Student\\
            10& Platform requests to insert the Team's name\\
            11& Student inserts the Team's name\\
            12& Platform sends confirmation e-mail to Student\\
            13& Platform gives visual feedback on web page of effective subscription\\
        \end{tabu} \\
        Exit Condition & Student is subscribed to Tournament and knows it\\
        Exception & Subscription deadline is passed, Student cannot subscribe to Tournament\\
        \specialReqLabel & - \\ 
    \end{tabu}
\end{center}
\useSvgWithCaption{Images/UML/sequenceDiagram/sequenceDiagramUC3.svg}{1.0}{1.0}{UC3 Sequence Diagram - Student subscribes to the Tournament}
\clearpage
\paragraph*{UC4 - User invites other user to collaborate} \label{uc:uc4}  
\begin{center}
    \begin{tabu}{|X[.2]|X|} \hline \everyrow{\hline}
        Name & User invites other user to collaborate \\ 
        Actors & Inviting user, invited user \\ 
        Entry Condition & Inviting user knows email of invited user \\ 
        Event Flow & \begin{tabu}{X X[50]}
            1& Inviting user clicks on "collaborate" button\\
            2& System shows inviting user a form to fill with the email of the user he wants to invite\\
            3& Inviting user clicks submit button\\
            4& System sends invite e-mail notification to invited user\\
            5& Invited user clicks on "Accept" button to accept invitation\\
            6& Invited user clicks on "Decline" button to decline invitation, or ignores it\\
        \end{tabu} \\
        Exit Condition & If invited user accepts invitation, invited user can collaborate with inviting one, else as nothing happened\\
        Exception & \begin{tabu}{X}
            Invited user e-mail address can be: non-existent, non-registered, of user of different role. In this case Inviting user get's notified of the error via e-mail.
        \end{tabu}  \\
        Special \newline Requirement & - \\ 
    \end{tabu}
\end{center}
"\nameref*{uc:uc4}" is a generalization of:\\
"\nameref{uc:uc4a}" and \\ "\nameref{uc:uc4b}".
\clearpage
\paragraph*{UC4a - Educator invites other educator to co-manage tournament} \label{uc:uc4a}
\begin{center}
    \begin{tabu}{|X[.2]|X|} \hline \everyrow{\hline}
        Name & Educator invites other educator to co-manage tournament\\ 
        Actors & Inviting educator, invited educator \\ 
        Entry Condition & Inviting educator is managing a tournament\\ 
        Event Flow & \begin{tabu}{X X[50]}
            1& Educator is in the managing page of the tournament\\
            & Event flow follows father's \nameref{uc:uc4} event flow
        \end{tabu} \\
        Exit Condition & Invited Educator can manage the tournament\\
        Exception & Same as \nameref{uc:uc4}\\
        Special \newline Requirement & - \\ 
    \end{tabu}
\end{center}
\useSvgWithCaption{Images/UML/sequenceDiagram/sequenceDiagramEducatorInvitesOtherEducator.svg}{1.0}{1.0}{UC4, UC4a Sequence Diagram - Educator invites educator to collaborate}
\clearpage
\paragraph*{UC4b - Student invites other student to participate in the tournament as a Team} \label{uc:uc4b}
\begin{center}
    \begin{tabu}{|X[.2]|X|} \hline \everyrow{\hline}
        Name & Student invites other student to participate in the tournament as a Team \\ 
        Actors & Inviting student, invited student\\ 
        Entry Condition & Inviting student is part of a team subscribed to a tournament \\ 
        Event Flow & \begin{tabu}{X X[50]}
            1& Student is in the tournament detail page\\
            & Follows follows father's \nameref{uc:uc4} event flow
        \end{tabu} \\
        Exit Condition & If invited student accepts invitation, invited user is part of inviting user's team\\
        Exception & Same as \nameref{uc:uc4}, with the verification of no participance to the Team in which he/she is invited or any other Team\\
        Special \newline Requirement & - \\ 
    \end{tabu}
\end{center}
\useSvgWithCaption{Images/UML/sequenceDiagram/sequenceDiagramStudentInvitesOtherStudent.svg}{1.0}{1.0}{UC4, UC4b Sequence Diagram - Student invites student to collaborate}
\clearpage
\begin{center}
    \begin{tabu}{|X[.2]|X|} \hline \everyrow{\hline}
        Name & Educator adds battle to tournament\\ 
        Actors & Educator \\ 
        Entry Condition & Educator has prepared: \newline 
        \begin{tabu}{X X[50]}
            -& Repository to be available for teams to fork\\
            -& Repository containing evaluation test suite\\
        \end{tabu} \newline 
        Tournament is not started\\ 
        Event Flow & \begin{tabu}{X X[50]}
            1& Educator is in tournament management page\\
            2& Educator clicks on "Add battle" button\\
            3& Platform shows a battle creation form\\
            4& Educator fills the form with link to repository in RMP\\
            5& Educator clicks on "Submit" button\\
            6& Platform registers battle addition to tournament\\
            7& Platfrom sends confirmation e-mail to Educator\\
        \end{tabu} \\
        Exit Condition & Platform is ready to test battles and code is available for students to fork\\
        Exception & Wrong  battle creation is discarded and Educator notified via e-mail of error\\
        Special Requirement & - \\ 
    \end{tabu}
\end{center}
\paragraph*{UC6 - Educator adds Battle to Tournament} \label{uc:uc6}
\begin{center}
    \begin{tabu}{|X[.2]|X|} \hline \everyrow{\hline}
        Name & Educator adds Battle to Tournament\\ 
        Actors & Educator \\ 
        Entry Condition & 
        Educator manages a Tournament \newline 
        Tournament is in "CREATED" state \newline 
        Educator has prepared: \newline 
        \begin{tabu}{X X[50]}
            -& Repository to be available for Teams to fork\\
            -& Repository containing evaluation test suite\\
        \end{tabu} \\
        Event Flow & \begin{tabu}{X X[50]}
            1& Educator is in Tournament management page\\
            2& Educator clicks on "Add Battle" button\\
            3& Platform shows a Battle creation form\\
            4& Educator fills the form with link to repository in RMP\\
            5& Educator clicks on "Submit" button\\
            6& Platform registers Battle addition to Tournament\\
            7& Platform sends confirmation e-mail to  the Educator\\
            8& Platform shows the new Battle's homepage to the Educator\\
        \end{tabu} \\
        Exit Condition & Platform is ready to test Battles and code is available for Students to fork\\
        Exception & - \\
        \specialReqLabel & - \\ 
    \end{tabu}
\end{center}
\useSvgWithCaption{Images/UML/sequenceDiagram/sequenceDiagramUC6.svg}{1.0}{1.0}{UC6 Sequence Diagram - Educator adds a Battle to the Tournament}
\clearpage
\paragraph*{UC7 - Educator creates Badge} \label{uc:uc7}
\begin{center}
    \begin{tabu}{|X[.2]|X|} \hline \everyrow{\hline}
        Name & Educator creates Badge \\ 
        Actors & Educator \\ 
        Entry Condition & Educator is managing a tournament \\ 
        Event Flow & \begin{tabu}{X X[50]}
            1& Educator is in tournament management page\\
            2& Educator clicks on "Add badge" button\\
            3& Platform shows badge creation form to educator\\
            4& Educator fills the form with badge image, description and test script\\
            5& Educator clicks on "Submit" button\\
            6& Platform checks badge test correctness\\
            7& Platform registers badge creation\\
            8& Platform sends a confirmation e-mail to educator\\
        \end{tabu} \\
        Exit Condition & Every user is able to see the existence of the badge in the tournament detail page\\
        Exception & If test script isn't correct, badge information is dropped and a notification e-mail is sent to the educator describing the issue\\
        Special \newline Requirement & - \\ 
    \end{tabu}
\end{center}
\useSvgWithCaption{Images/UML/sequenceDiagram/sequenceDiagramBadges.svg}{1.0}{1.0}{UC7 Sequence Diagram - Educator creates Badge}
\clearpage
\paragraph*{UC8 - Student gets awarded with Badge}   \label{uc:uc8}
\begin{center}
    \begin{tabu}{|X[.2]|X|} \hline \everyrow{\hline}
        Name & Student gets awarded with Badge\\ 
        Actors & Student \\ 
        Entry Condition & \begin{tabu}{@{}X}
            Student is part of a Team, and has contributed to Team's repository \\
            Tournament is in "OFFLINE" state\\
        \end{tabu} \\
        Event Flow & \begin{tabu}{X X[50]}
            1& System runs Badge tests on all repositories\\
            2& Badge tests give a list of RMP handles and e-mail addresses to which assign corresponding Badge\\
            3& System searches for corresponding accounts\\
            4& System assigns Badges to each User found\\
            5& System sends e-mail notification to chosen Students\\
        \end{tabu} \\
        Exit Condition & Random Platform User can see Badge in the specific Student profile page\\
        Exception & - \\
        \specialReqLabel & Commit User data is consistent with CKB Platform User data \\ 
    \end{tabu}
    \tableEntryByLabel{uc:uc8}
\end{center}
\useSvgWithCaption{Images/UML/sequenceDiagram/sequenceDiagramUC8.svg}{1.0}{1.0}{UC8 Sequence Diagram - Student gets awarded with Badge}
\clearpage
%\paragraph*{UC6 - Educator adds Battle to Tournament} \label{uc:uc6}
\begin{center}
    \begin{tabu}{|X[.2]|X|} \hline \everyrow{\hline}
        Name & Educator adds Battle to Tournament\\ 
        Actors & Educator \\ 
        Entry Condition & 
        Educator manages a Tournament \newline 
        Tournament is in "CREATED" state \newline 
        Educator has prepared: \newline 
        \begin{tabu}{X X[50]}
            -& Repository to be available for Teams to fork\\
            -& Repository containing evaluation test suite\\
        \end{tabu} \\
        Event Flow & \begin{tabu}{X X[50]}
            1& Educator is in Tournament management page\\
            2& Educator clicks on "Add Battle" button\\
            3& Platform shows a Battle creation form\\
            4& Educator fills the form with link to repository in RMP\\
            5& Educator clicks on "Submit" button\\
            6& Platform registers Battle addition to Tournament\\
            7& Platform sends confirmation e-mail to  the Educator\\
            8& Platform shows the new Battle's homepage to the Educator\\
        \end{tabu} \\
        Exit Condition & Platform is ready to test Battles and code is available for Students to fork\\
        Exception & - \\
        \specialReqLabel & - \\ 
    \end{tabu}
\end{center}
\useSvgWithCaption{Images/UML/sequenceDiagram/sequenceDiagramUC6.svg}{1.0}{1.0}{UC6 Sequence Diagram - Educator adds a Battle to the Tournament}
\clearpage
%\paragraph*{UC6 - Educator adds Battle to Tournament} \label{uc:uc6}
\begin{center}
    \begin{tabu}{|X[.2]|X|} \hline \everyrow{\hline}
        Name & Educator adds Battle to Tournament\\ 
        Actors & Educator \\ 
        Entry Condition & 
        Educator manages a Tournament \newline 
        Tournament is in "CREATED" state \newline 
        Educator has prepared: \newline 
        \begin{tabu}{X X[50]}
            -& Repository to be available for Teams to fork\\
            -& Repository containing evaluation test suite\\
        \end{tabu} \\
        Event Flow & \begin{tabu}{X X[50]}
            1& Educator is in Tournament management page\\
            2& Educator clicks on "Add Battle" button\\
            3& Platform shows a Battle creation form\\
            4& Educator fills the form with link to repository in RMP\\
            5& Educator clicks on "Submit" button\\
            6& Platform registers Battle addition to Tournament\\
            7& Platform sends confirmation e-mail to  the Educator\\
            8& Platform shows the new Battle's homepage to the Educator\\
        \end{tabu} \\
        Exit Condition & Platform is ready to test Battles and code is available for Students to fork\\
        Exception & - \\
        \specialReqLabel & - \\ 
    \end{tabu}
\end{center}
\useSvgWithCaption{Images/UML/sequenceDiagram/sequenceDiagramUC6.svg}{1.0}{1.0}{UC6 Sequence Diagram - Educator adds a Battle to the Tournament}
\clearpage

\subsection{Performance Requirements}
The platform, that can be described as a Web App, has to be able to manage multiple Users requests, constantly update rankings for related Tournaments, perform tests on provided code and assign related Score, all the same time. 

So, in order to satisfy all Requirements defined above, providing a good time response for each task, the Platform should have good time performance. Being a Web App, this implies a high concurrent programming implementation, to 
manage multiple requests at the same time, and a rapid related Data Storage System. 

\subsection{Design Constraints}

\subsubsection{Standard Compliance}
First of all, the platform has to respect privacy standards, such as GDPR for EU countries or similar regulations, in order to protect personal information provided by Users when they sign in, their code and identity across the Web App.

It is also important to respect internet protocols standards and their rules, in order to make the app run and communicate properly. Other standards are related to UI accessibility features and servers' power consumption. 

\subsubsection{Hardware Limitations}
Hardware limitations in the RASD context are referred to User's machine capabilities and server's limits. 

In fact, the devices used by both Students and Educators are constrained by memory capacity, CPU capabilities and GPU computational power. The first is essential to memorize the file to upload in RMP, run the software to access internet
 and to manage the Web App data itself. The second is fundamental to run the connection tasks and the site, receiving and elaborating inputs for the User and CKB's responses. Finally, GPU is essential to render the platform UI.\\
\\
 On server side, instead, is important for our App to have an organized storage to preserve all necessary data and enough computational power in our CPUs to manage multiple requests for each machine that composes CKB's server. 
 
In fact, to satisfy all possible inputs from all possible Users, formulating for each one a response, and meet requirements, is critical to have a multiple machine architecture on our servers. 

To complete, a multiple internet connection ports is fundamental, to make the platform available on the internet and make it capable of communicate with Users and RMPs.
\subsubsection{Any other constraint}

\subsection{Software System Attributes}

\subsubsection{Reliability}
The platform described in this document should be an always available Web App. 

Downtimes are meant to be as less as possible, and the overall system, in order to prevent evaluation, ranking, registration, search problems, should be designed with robustness in mind and with built-in mechanisms to manage situations 
that could compromise information, materials and commands assigned to the platform. All of this performed transparently to the Users as much as possible.

\subsubsection{Availability}
The platform should be able as much as possible, 24/7, 365 days per year. 

The minimum availability rate should be 99\%. An inferior percentage could compromise educational activities potentially damaging Students.

\subsubsection{Security}
Both Users' devices and servers communications must always take place only via encrypted channels, using appropriate cryptographic protocols such as SSL. 

All operations and contributions to the platform, except automated tasks performed on server side communicating with RMP, must be always explicitly approved by the Users.

\subsubsection{Maintainability}
In order to extend as much as possible the platform's life cycle it should be designed with modularity in mind. 

This approach will guarantee the interchanging of old modules to update or to substitute, possibly maintaining the same interfaces and core functionalities, at minimum cost.

\subsubsection{Portability}
The platform intended to be build would be used independently by the OS provided with the machine. 

In fact CKB can be accessed simply using just a web browser, on which compatibility the developers should concentrate.\\ 
\\
On User side, it would be sufficient to guarantee portability to build the Web App using common internet standards and protocols to make it usable on any web browser.\\
\\
About RPM portability, the platform should be provided with the right interfaces and modules to communicate with the largest amount of RMPs available, in order to acquire and run code evaluating it.