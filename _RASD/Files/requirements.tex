\subsection{External Interface Requirements}

\subsubsection{User Interfaces}
In order to access the platform the crucial interface needed by the User is the one provided by a web browser. In fact, it is sufficient to reach the CKB's URL to start with log in or sign up operations, described in the scenarios, or, 
after authentication, interact with the platform. The platform software, in fact, is a Web App. Instead, to join a Battle a RMP link would be required.

\subsubsection{Hardware Interfaces}
Users have to provide themselves with a device able to access internet. It would be sufficient that it is equipped with a Wi-Fi and/or Ethernet interface. Of course would be crucial that it provides adequate components to allow Users 
to interact with the platform, showing its interfaces.

\subsubsection{Software Interfaces}
As defined above, a web browser is the only software needed to access the platform. The interfaces that have to be supported are the ones defined by the Web Page rendering. About RMP, the platform software run on the server would be 
equipped with appropriate interfaces to interact with the RMP provided by the Student or the Team in the steps described in previous sections of this document.
\subsubsection{Communication Interfaces}
Communication Interfaces needed are the one necessary to access the internet. So for the User, TCP and HTTP interfaces would be crucial to reach the server on which the platform runs, while for the CKB's app it is fundamental to 
interact with RMP.

\newpage

\subsection{Functional Requirements}
Here follows a list of the platform functional requirements:
\begin{enumerate}[label=$\bullet$ \textbf{R\arabic*:}]
    \item The platform allows Users to register to the platform itself either as Student either Educator.
    \item The platform allows registered Students to join a Battle.
    \item The platform allows registered Students to form Teams.
    \item The platform allows registered Students to invite other Students to join a Team.
    \item The platform allows registered Educators to create Tournaments.
    \item The platform allows registered Educators to create Battles.
    \item The platform allows registered Educators to create Badges.
    \item The platform assigns a Battle Score to Teams' work.
    \item The platform provides a Teams' ranking based on the Tournament Score within a Tournament context.
    \item The platform allows registered Educators to add other Educators to a Tournament.
    \item The platform allows Users to search for other Users.
    \item The platform interacts properly with different RMPs to acquire the latest versions of code uploaded by related Teams. 
    \item The platform should be able to manage multiple requests at the same time
\end{enumerate}

\newpage

\subsubsection{Use case diagrams}
\useSvgWithCaption{./Images/UML/useCaseDiagram/useCase1.svg}{0.95}{0.95}{Use case diagram of the login and the invitation}
\useSvgWithCaption{./Images/UML/useCaseDiagram/useCase2.svg}{0.93}{0.93}{Use case diagram of the Battle and the Tournament}

\newpage

\subsubsection{Use cases and associated sequence diagrams}
Here follow Use Case tables followed by respective sequence diagrams.
\paragraph*{UC1}
\paragraph{UC1 - User signs in to the Platform} \label{uc:uc1} 
\begin{center}
    \begin{tabu}{|X[.2]|X|} \hline \everyrow{\hline} 
        Name & User signs in to the Platform \\
        Actors & User \\ 
        Entry Condition & User has a valid e-mail address and valid RMP handle\\ 
        Event Flow & \begin{tabu}{X X[50]}
            1& At Homepage click on "Sign in" button\\
            2& System shows User the registration form\\
            3& User fills form with data, caring to choose his role accordingly\\
            4& User clicks on "Submit" button \\
            5& Platform saves submitted information\\
            6& Platform sends e-mail confirmation link to User\\
            7& User confirms e-mail by clicking confirmation e-mail\\
        \end{tabu} \\
        Exit Condition & User correctly registered in the Platform, User needs to login to use the Platform\\
        Exception & User provides an Email already in use.\\
        \specialReqLabel & - \\ 
    \end{tabu}
    \tableEntryByLabel{uc:uc1}
\end{center} 
"\nameref*{uc:uc1}" is a generalization of:\\
"\nameref{uc:uc1a}" and \\ "\nameref{uc:uc1b}".
\clearpage
\paragraph*{UC1a - Student signs in to the Platform} \label{uc:uc1a} 
\begin{center}
    \begin{tabu}{|X[.2]|X|} \hline \everyrow{\hline} 
        Name & Student signs in to the Platform \\ 
        Actors & Student \\ 
        Entry Condition & Same as \nameref{uc:uc1} \\ 
        Event Flow & Same as \nameref{uc:uc1} with exception of \newline \begin{tabu}{X X[50]}
            3& Student chooses Student role while filling in form\\
        \end{tabu} \\
        \exitCondLabel & Same as \nameref{uc:uc1}\\
        Exception & Same as \nameref{uc:uc1}\\
        \specialReqLabel & Same as \nameref{uc:uc1}\\ 
    \end{tabu}
    \tableEntryByLabel{uc:uc1a}
\end{center}
\useSvgWithCaption{./Images/UML/sequenceDiagram/sequenceDiagramUC1a.svg}{1.0}{1.0}{UC1, UC1a, Sequence Diagram - Student signs in to the Platform}
\clearpage

\paragraph*{UC1b - Educator signs in to the Platform} \label{uc:uc1b} 
\begin{center}
    \begin{tabu}{|X[.2]|X|} \hline \everyrow{\hline}
        Name & Educator signs in to the Platform \\ 
        Actors & Educator \\ 
        Entry Condition & Same as \nameref{uc:uc1} \\ 
        Event Flow & Same as \nameref{uc:uc1} with exception of \newline \begin{tabu}{X X[50]}
            3& Educator chooses Educator role while filling in form\\
        \end{tabu} \\
        Exit Condition & Same as \nameref{uc:uc1}\\
        Exception & Same as \nameref{uc:uc1}\\
        \specialReqLabel & Same as \nameref{uc:uc1}\\ 
    \end{tabu}
    \tableEntryByLabel{uc:uc1b}
\end{center}

\useSvgWithCaption{./Images/UML/sequenceDiagram/sequenceDiagramUC1b.svg}{1.0}{1.0}{UC1, UC1b Sequence Diagram - Educator signs in to the Platform}
\clearpage
\paragraph*{UC2}
\paragraph*{UC2 - User logs in to the Platform} \label{uc:uc2}
\begin{center}
    \begin{tabu}{|X[.2]|X|} \hline \everyrow{\hline}
        Name & User logs in to the Platform \\ 
        Actors & User \\ 
        Entry Condition & User has valid credentials corresponding to an account \\ 
        Event Flow & \begin{tabu}{X X[50]}
            1& User is on web app welcome page\\
            2& User clicks on "Log in" button\\
            3& Platform shows User login form\\
            4& User fills the form accordingly\\
            5& User clicks on login button\\
            6& Platform shows User's homepage\\
        \end{tabu} \\
        Exit Condition & User finds in a valid session in his homepage free to do whatever he/she can do in the Platform\\
        Exception & User fills form with wrong User data, Platform visually notifies about the error\\
        \specialReqLabel & -\\ 
    \end{tabu}
\end{center}
\useSvgWithCaption{./Images/UML/sequenceDiagram/sequenceDiagramUC2.svg}{0.75}{0.75}{UC2 Sequence Diagram - User logs in to the Platform}
\clearpage
\paragraph*{UC3}
\begin{center}
    \begin{tabu}{|X[.2]|X|} \hline \everyrow{\hline}
        Name & User invites other user to collaborate \\ 
        Actors & Inviting user, invited user \\ 
        Entry Condition & Inviting user knows email of invited user \\ 
        Event Flow & \begin{tabu}{X X[50]}
            1& Inviting user clicks on "collaborate" button\\
            2& System shows inviting user a form to fill with the email of the user he wants to invite\\
            3& Inviting user clicks submit button\\
            4& System sends invite e-mail notification to invited user\\
            5& Invited user clicks on "Accept" button to accept invitation\\
            6& Invited user clicks on "Decline" button to decline invitation, or ignores it\\
        \end{tabu} \\
        Exit Condition & If invited user accepts invitation, invited user can collaborate with inviting one, else as nothing happened\\
        Exception & \begin{tabu}{X}
            Invited user e-mail address can be: non existent, non registered, of user of different role. In this case Inviting user get's notified of the error via e-mail.
        \end{tabu}  \\
        Special \newline Requirement & - \\ 
    \end{tabu}
\end{center}

\paragraph*{UC3a}
\begin{center}
    \begin{tabu}{|X[.2]|X|} \hline \everyrow{\hline}
        Name & Educator invites other educator to co-manage tournament\\ 
        Actors & Inviting educator, invited educator \\ 
        Entry Condition & Inviting educator is managing a tournament\\ 
        Event Flow & \begin{tabu}{X X[50]}
            1& Educator is in the managing page of the tournament\\
            & Event flow follows father use case
        \end{tabu} \\
        Exit Condition & Invited Educator can manage the tournament\\
        Exception & Same as UC3\\
        Special \newline Requirement & - \\ 
    \end{tabu}
\end{center}
\useSvgWithCaption{Images/UML/sequenceDiagram/sequenceDiagramEducatorInvitesOtherEducator.svg}{1.0}{1.0}{UC3a Sequence Diagram - Educator invites educator to collaborate}

\paragraph*{UC3b}
\begin{center}
    \begin{tabu}{|X[.2]|X|} \hline \everyrow{\hline}
        Name & Student invites other student to participate in the tournament as a Team \\ 
        Actors & Inviting student, invited student\\ 
        Entry Condition & Inviting student is part of a team subscribed to a tournament \\ 
        Event Flow & \begin{tabu}{X X[50]}
            1& Student is in the tournament detail page\\
            & Follows UC3's event flow
        \end{tabu} \\
        Exit Condition & If incited student accepts invitation, invited user is part of inviting user's team\\
        Exception & Same as UC3, with the verification of no participance to the Team in which he/she is invited or any other Team\\
        Special \newline Requirement & - \\ 
    \end{tabu}
\end{center}
\useSvgWithCaption{Images/UML/sequenceDiagram/sequenceDiagramStudentInvitesOtherStudent.svg}{1.0}{1.0}{UC3b Sequence Diagram - Student invites student to collaborate}
\clearpage
\paragraph*{UC4 - User invites other user to collaborate} \label{uc:uc4}  
\begin{center}
    \begin{tabu}{|X[.2]|X|} \hline \everyrow{\hline}
        Name & User invites other user to collaborate \\ 
        Actors & Inviting user, invited user \\ 
        Entry Condition & \begin{tabu}{@{}X}
            Inviting user knows email of invited user \\ 
            Tournament is in the right state to invite user\\
        \end{tabu} \\
        Event Flow & \begin{tabu}{X X[50]}
            1& Inviting user clicks on "collaborate" button\\
            2& System shows inviting user a form to fill with the email of the user he wants to invite\\
            3& Inviting user clicks submit button\\
            4& System sends invite e-mail notification to invited user\\
            5& Invited user clicks on "Accept" button to accept invitation\\
            6& Invited user clicks on "Decline" button to decline invitation, or ignores it\\
        \end{tabu} \\
        Exit Condition & If invited user accepts invitation, invited user can collaborate with inviting one, else as nothing happened\\
        Exception & \begin{tabu}{@{}X}
            Invited user e-mail address can be: non-existent, non-registered, of user of \\different role. In this case Inviting user get's notified of the error via e-mail.
        \end{tabu}  \\
        \specialReqLabel & - \\ 
    \end{tabu}
\end{center}
"\nameref*{uc:uc4}" is a generalization of:\\
"\nameref{uc:uc4a}" and \\ "\nameref{uc:uc4b}".
\clearpage
\paragraph*{UC4a - Educator invites other educator to co-manage tournament} \label{uc:uc4a}
\begin{center}
    \begin{tabu}{|X[.2]|X|} \hline \everyrow{\hline}
        Name & Educator invites other educator to co-manage tournament\\ 
        Actors & Inviting educator, invited educator \\ 
        Entry Condition & \begin{tabu}{@{}X}
            Same as \nameref{uc:uc4} \\
            Inviting educator is managing a tournament\\ 
            Tournament is in "CREATED" state\\
        \end{tabu} \\
        Event Flow & \begin{tabu}{X X[50]}
            1& Educator is in the managing page of the tournament\\
            & Event flow follows father's \nameref{uc:uc4} event flow
        \end{tabu} \\
        Exit Condition & Invited Educator can manage the tournament\\
        Exception & Same as \nameref{uc:uc4}\\
        \specialReqLabel & - \\ 
    \end{tabu}
\end{center}
\useSvgWithCaption{Images/UML/sequenceDiagram/sequenceDiagramUC4a.svg}{1.0}{1.0}{UC4, UC4a Sequence Diagram - Educator invites educator to collaborate}
\clearpage
\paragraph*{UC4b - Student invites other student to participate in the tournament as a Team} \label{uc:uc4b}
\begin{center}
    \begin{tabu}{|X[.2]|X|} \hline \everyrow{\hline}
        Name & Student invites other student to participate in the tournament as a Team \\ 
        Actors & Inviting student, invited student\\ 
        Entry Condition & \begin{tabu}{@{}X}
            Same as \nameref{uc:uc4} \\
            Inviting student is part of a team subscribed to a tournament \\ 
            Tournament is in "ONLINE" state\\
        \end{tabu} \\
        Event Flow & \begin{tabu}{X X[50]}
            1& Student is in the tournament detail page\\
            & Event flow follows father's \nameref{uc:uc4} event flow
        \end{tabu} \\
        Exit Condition & If invited student accepts invitation, invited user is part of inviting user's team\\
        Exception & Same as \nameref{uc:uc4}, with the verification of no participance to the Team in which he/she is invited or any other Team\\
        \specialReqLabel & - \\ 
    \end{tabu}
\end{center}
\useSvgWithCaption{Images/UML/sequenceDiagram/sequenceDiagramUC4b.svg}{1.0}{1.0}{UC4, UC4b Sequence Diagram - Student invites student to collaborate}
\clearpage
\paragraph*{UC5 - Educator creates Tournament} \label{uc:uc5}
\begin{center}
    \begin{tabu}{|X[.2]|X|} \hline \everyrow{\hline}
        Name & Educator creates Tournament \\ 
        Actors & Educator\\ 
        Entry Condition & Educator has a clear plan of how to run the tournament\\ 
        Event Flow & \begin{tabu}{X X[50]}
            1& Educator is in its homepage\\
            2& Educator clicks on "Create tournament" button\\
            3& Platform shows tournament creation form to the educator\\
            4& Educator fills in the form\\
            5& Educator clicks on "Submit" button \\
            6& Platform registers tournament creation \\
            7& Platform sends e-mail notification of success to educator \\
            8& Platform shows the new tournament's homepage to the educator\\
            9& Platform sends e-mail notification of tournament existence to students \\
        \end{tabu} \\
        Exit Condition & Tournament is created and available for students to subscribe to it\\
        Exception & -\\
        Special \newline Requirement & - \\ 
    \end{tabu}
\end{center}

\useSvgWithCaption{Images/UML/sequenceDiagram/sequenceDiagramEducatorCreatesTournament.svg}{1.0}{1.0}{UC5 Sequence Diagram - Educator creates tournament}
\clearpage

\subsection{Performance Requirements}
The platfrom, that can be described as a Web App, has to be able to manage multiple Users requests, constantly update rankings for related Tournaments, perform tests on provided code and assign related Score, all the same time. So, in order to satisfy all Requirements defined above, providing a good time response for each tasks, the Platform should have good time performance. Being a Web App, this implies a high concurrent programming implementation, to manage multiple requests at the same time, and a rapid related Data Storage System. 
\subsection{Design Constraints}

\subsubsection{Standard Compliance}
First of all, the platform has to respect privacy standards, sucha as GDPR for EU countries or similar regulations, in order to protect personal informations provided by Users when they sign in, their code and identity across the Web App. It is also important to respect internet protocols standards and their rules, in order to make the app run and comunicate properly. Other standards are related to UI accessibility features and servers' power consumption. 
\subsubsection{Hardware Limitations}
Hardware limitations in the RASD context are refered to User's machine capabilities and server's limits. In fact, the devices used by both Students and Educators are constrained by memory capacity, CPU capabilities and GPU computational power. The first is essential to memorize the file to upload in RMP, run the software to access internet and to manage the Web App data itself. The second is fundamental to run the connection tasks and the site, receiving and elaborating inputs for the User and CKB's responses. Finally GPU is essential to render the platform UI. On server side, instead, is important for our App to have an organized storage to preserve all necessary data and enough computational power in our CPUs to manage multiple requests for each machine that compones CKB's server. In fact, to satify all possibile inputs from all possibile Users, formulating for each one a response, and meet requirements, is critical to have a multiple machine architecure on our servers. To complete, a multiple internet connection ports is fundamental, to make the platform available on the internet and make it capable of comunicate with Users and RMPs.
\subsubsection{Any other constraint}

\subsection{Software System Attributes}

\subsubsection{Reliability}
The platform described in this document should be an always available Web App. Downtimes are meant to be as less as possible, and the overall system, in order to prevent evaluation, ranking, registration, search problems, should be designed with robustness in mind and with built in mechanisms to manage situations that could compromise informations, materials and commands assigned to the platform. All of this performed in a trasparent way to the Users as much as possibile.
\subsubsection{Availability}
The platform should be able as much as possible, 24/7, 365 days per year. The minimum availability rate shuould be 99\%. An inferior percentage could compromise educational activities potentially damaging students.
\subsubsection{Security}
Both Users' devices and servers comunications must always take place only via encrypted channels, using appropriate criptographic protolocs such as SSL. All operations and contributions to the platform, with the exception of automated tasks performed on server side comunicating with RMP, must be always explicitly approved by the Users.
\subsubsection{Maintainability}
In order to extend as much as possible the platform's life cycle it should be designed with modularity in mind. This approach will guarantee the interchanging of old modules to update or to substitute, possibly maintaing the same interfaces and core functionalities, at minimum cost.
\subsubsection{Portability}
The platform intended to be build would be used independetly by the OS provided with the machine. In fact CKB can be accesed simply using just a web browser, on which compatibility the developers should concentrate. On User side, it would be sufficient to guarantee portability to build the Web App using common internet standards and protocols to make it usable on any web broswer. About RPM portability, the platform should be provided with the right interfaces and modules to comunicate with the largest amount of RMPs available, in order to acquire and run code evaulating it.





